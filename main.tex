\documentclass[12pt]{article}
\usepackage{blindtext}
\usepackage{hyperref}
\usepackage{amsmath,epsfig}
\usepackage{amsfonts}
\usepackage[T1]{fontenc}
\usepackage{titling}
\usepackage{commath}
\usepackage{bm}

\setlength{\droptitle}{-10em}
\title{Final report draft}

\begin{document}

\maketitle
\begin{abstract}

\end{abstract}

\section{Introduction}
	The goal of this project is to quantitatively compare the results of introducing importance sampling in a ray tracing model. Instead of picking a random direction for the next ray in the light path and calculating the BSDF, importance sampling lets the model to automatically select a direction based on how much it will contribute to the radiance, so the insignificant directions are selected less often, resulting in faster convergence rate of the image. In this project, we observe the convergence rate under direct sampling and importance sampling methods by comparing how fast the Frobenius norm of the difference between the converged image and an image generated given difference number of samples.
    
\section{Methods}

Let \(\bm o\) be the unit outbound light direction vector, and let \(\bm i\) be the unit inbound light direction vector.
Let \(\bm n\) be the normal vector.

I am using the Fresnel formulation from PBRT.
\begin{align*}
  r_{\parallel}^2 &= \frac{\del{\eta^2 + \kappa^2}\cos^2 \theta_{\bm i} - 2\eta \cos \theta_{\bm i} + 1}{\del{\eta^2 + \kappa^2}\cos^2 \theta_{\bm i} + 2\eta \cos \theta_{\bm i} + 1}\\
  r_{\perp}^2 &= \frac{\del{\eta^2 + \kappa^2} - 2\eta \cos \theta_{\bm i} + \cos^2 \theta_{\bm i}}{\del{\eta^2 + \kappa^2} + 2\eta \cos \theta_{\bm i} + \cos^2 \theta_{\bm i}}
\end{align*}

\section{Results}

\section{Discussion}

\end{document}
